

\documentclass{kulakarticle}

\usepackage[dutch]{babel}
\usepackage[utf8]{inputenc}
\usepackage{lipsum}
\usepackage{amsmath, amssymb, amsthm}
\usepackage{pdfpages}


\title{Biologische data analyse app}
\author{Marthe Böting, Robin Bruneel, Toon Ingelaere}
\date{\today}

\begin{document}
	\maketitle
	\section*{Biologische data analyse app}
	\section{Inleiding}
	Als gevolg van een bloedklonter in een hersenbloedvat ontstaat een ischemische beroerte. Deze klonter moet men zo snel mogelijk verwijderen dit kan met behulp van een geneesmiddel die de klonter oplost of door middel van mechanische verwijdering. Door deze mechanische verwijdering kan men de bloedklonter verder analyseren in het labo om een duidelijker beeld te krijgen.
	\newline
	Vandaag de dag verliest men zeer veel tijd aan het verwerken van de bloedklonters. Er is ons dan ook gevraagd om een app te ontwikkelen die de analyse van de foto’s volledig automatisch kan uitvoeren.
	\newline
	In dit verslag gaan we eerst in op wat de klant specifiek van ons verwacht en aan welke specificaties ons ontwerp moet voldoen. Hierna gaan we ons design bespreken en toelichten. Verder bespreken we ook onze voorlopige resultaten. Ten slotte wordt er nog een blik geworpen naar de vakken uit eerste 3 semesters die ons hierbij geholpen hebben.
	
	\section{Klantenvereisten}
	De klant verwacht een gebruiksvriendelijke app waarbij hij een foto kan ingeven en dat deze automatisch bewerkt wordt. De foto moet  worden bijgesneden en de achtergrond moet verwijderd worden. Daarnaast is het de bedoeling om de bloedklonter te analyseren met andere woorden het berekenen van het percentage van een eiwit. Dit gebeurt door een hoeveelheid kleur te quantificeren in de bloedklonter. 

	\section{Ontwerpspecificatie}
	De klant wil een app die elke foto kan bijwerken en ook een percentage van eiwitten terug geeft. Er zijn geen verdere voorwaarden verbonden aan de grootte van de foto. Om de foto bij te snijden moet er rekening mee gehouden worden dat de bloedklonter er nog volledig opstaat. Desondanks mag de rand niet te groot zijn omdat dit anders te veel geheugen in beslag neemt. De achtergrond moet volledig wit gemaakt worden zodat er geen fouten worden gemaakt bij het berekenen van een hoeveelheid kleur. De app moet voor iedereen bruikbaar zijn. Daarnaast moeten de verschillende foto's over de verschillende fasen getoond worden. 

	
	\section{Integratie met vakken uit eerste 3 semsters}
	Om dit probleem op te lossen hebben we gebruik gemaakt van matlab. Het is dan ook een meerwaarde dat we in het eerste semester 'beginselen van programmeren' gehad hebben waardoor we sneller vertrouwd geraken met deze programmeertaal. Daarnaast leren we in 'nummerieke wiskunde' ook werken met matlab. Statistische analyse gebruiken we voor het analyseren van de verschillende kleuren.
	
	\section{Voorlopige conclusie} 
	Momenteel zitten we op schema. De achtergrond kunnen we al verwijderen , we moeten juist nog een methode bedenken om de ... Daarnaast moeten we ons verder focussen om de figuur in te kleuren en om de app verder af te werken.
	
	
\end{document}